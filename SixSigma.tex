\documentclass{abnt}
%Arquivo com os principais pacotes usados e suas descrições.

%%%%%%%%%%%%%%%%%%%%%%%%%%%%%%%%%%%%%%%%%
% 			Idiomas e Acentos			%
%%%%%%%%%%%%%%%%%%%%%%%%%%%%%%%%%%%%%%%%%
\usepackage[brazil]{babel} % Habilita o uso do idioma português do brasil (PT-BR).
\usepackage[T1]{fontenc} 
%\usepackage{fontspec} % Habilita maior variedade de acentos. Pode ser necessario adicionar outros pacotes.
\usepackage{lmodern} % Habilita o uso da font Latin Modern.


%%%%%%%%%%%%%%%%%%%%%%%%%%%%%%%%%%%%%%%%%
% 				TABELAS					%
%%%%%%%%%%%%%%%%%%%%%%%%%%%%%%%%%%%%%%%%%
\usepackage{tabulary} % Cria tabelas mais facilmente.
\usepackage{booktabs} % Melhora o visual das tabelas.
\usepackage{table}{xcolor} % Pacote de cor pra as tabelas.
\usepackage{caption} % Melhora as legendas de imagens, tabela etc.

%%%%%%%%%%%%%%%%%%%%%%%%%%%%%%%%%%%%%%%%%
% 				IMAGENS					%
%%%%%%%%%%%%%%%%%%%%%%%%%%%%%%%%%%%%%%%%%
%\usepackage{graphicx} % Facilita a inserção de imagens.


%%%%%%%%%%%%%%%%%%%%%%%%%%%%%%%%%%%%%%%%%
% 			CÓDIGO FONTE				%
%%%%%%%%%%%%%%%%%%%%%%%%%%%%%%%%%%%%%%%%%
%Documentação de código fonte.
\usepackage{listings}


%%%%%%%%%%%%%%%%%%%%%%%%%%%%%%%%%%%%%%%%%
% 	Símbolos e Caracteres Matemáticos	%
%%%%%%%%%%%%%%%%%%%%%%%%%%%%%%%%%%%%%%%%%
\usepackage{amsmath}
\usepackage{amssymb}
\usepackage{amsfonts}
%\usepackage{mathspec} %Habilita o uso das fontes e dos caracteres matematicos.


%%%%%%%%%%%%%%%%%%%%%%%%%%%%%%%%%%%%%%%%%
%				ABNT					%
%%%%%%%%%%%%%%%%%%%%%%%%%%%%%%%%%%%%%%%%%
%\usepackage[alf]{abntcite} % Ordena as referencias em ordem alfabética.
\usepackage{url} %Facilita o uso de url. Pode-se usar o comando \url{...}.


%%%%%%%%%%%%%%%%%%%%%%%%%%%%%%%%%%%%%%%%%
% 			Configurações				%
%%%%%%%%%%%%%%%%%%%%%%%%%%%%%%%%%%%%%%%%%
\captionsetup{justification=centering,labelfont=bf} %Formata a legenda das figuras.
%\graphicspath{{../imgs/}} %Define o diretorio padrão para buscar as imagens da apresentação.  
%\setromanfont[Ligatures=TeX]{Crimson}
%\defaultfontfeatures{Scale=MatchLowercase, Mapping=tex-tex}

%%%%%%%%%%%%%%%%%%%%%%%%%%%%%%%%%%%%%%%%%
%				BEAMER					%
%%%%%%%%%%%%%%%%%%%%%%%%%%%%%%%%%%%%%%%%%
%Define algumas configurações que serão validas para todo o documento.  
\setbeamertemplate{section in toc}[sections numbered]
\setbeamertemplate{subsection in toc}[subsections numbered]
\setbeamertemplate{background canvas}[vertical shading][bottom=blue!3,top=blue!7]
\setbeamertemplate{caption}[numbered]


\author{
	Matheus Liberato Domingues da Silva, \texttt{matheusliberatosbs@gmail.com} \and
	Guilherme Augusto de Macedo, \texttt{gaugustomacedo@gmail.com} \and
	Victor Hugo Carlquist da Silva, \texttt{victorcarlquist@gmail.com} \and
	Denis Fournier de Carvalho, \texttt{denao\_carvalho@hotmail.com} 
}

\title{ Sistema Seis Sigma de Produção}
\date{\today}

\begin{document}
	\maketitle
	\tableofcontents

	\begin{abstract}
		Your abstract goes here...
		...
	\end{abstract}
	
	\chapter {Introdução}
	
	\chapter {História}
	
	\chapter {Níveis de Sigma}
		\section {Introdução}
			O sigma é utilizado para medir a variância de qualquer processo. Os níveis de sigma medem o desempenho do processo de uma empresa.
			Geralmente uma empresa adota níveis 3 ou 4 do sigma, que são níveis considerados normais.
			
			O sigma ($\sigma$) é calculado pela seguinte fórmula: 
			\begin{center}
			    \begin{equation}
			       Z = \frac{x - \mu}{\sigma}
		        \end{equation}
		     x = ponto que se deseja converter em Z; $\mu$ =  média da normal original ; $\sigma$ = desvio padrão da normal original
			\end{center}
			
			 
		\section {Níveis de Qualidade}
		    Cada nível de sigma possuí um limite de desvios (problemas), que é medido em  \textit{Problemas por Milhão} (PPM) 
			\begin{itemize}
			    \item {1 Sigma}
			        \subitem O 1 Sigma tolera até 697700 (PPM), possuindo um fator de sucesso do processo de 30,23\%. 
			    \item {2 Sigma}
			        \subitem O 2 Sigma tolera até 308700 (PPM), possuindo um fator de sucesso do processo de 69,13\%.
			    \item {3 Sigma}
			        \subitem O 3 Sigma tolera até 66810 (PPM), possuindo um fator de sucesso do processo de 93,32\%.
			    \item {4 Sigma}
			        \subitem O 4 Sigma tolera até 6210 (PPM), possuindo um fator de sucesso do processo de 99,379\%.
			    \item {5 Sigma}
			        \subitem O 5 Sigma tolera até 233 (PPM), possuindo um fator de sucesso do processo de 99,9767\%.
			    \item {6 Sigma}
			        \subitem O 6 Sigma tolera até 3,4 (PPM), possuindo um fator de sucesso do processo de 99,99966\%.
			\end{itemize}
		\section {Curva Normal} 
		    
		\section {Vantagens do 6 Sigma}
				
	\chapter {Metodologias}
		\section {Introdução}
			\section {DMAIC}
				\subsection {Definir}
					\subsubsection {Objetivo}
					\subsubsection {Entendendo o processo}
					\subsubsection {Preparação para a próxima fase}
				\subsection {Medir}
					\subsubsection {Objetivo}
					\subsubsection {Mapeando processo}
					\subsubsection {Coletando os dados}
					\subsubsection {Analisando os dados coletados}
					\subsubsection {Calculando o nível de Sigma}
					\subsubsection {Preparação para a próxima fase}
				\subsection {Analisar}
					\subsubsection {Melhorar}
					\subsubsection {Objetivo}
					\subsubsection {Achando a causa raiz}
					\subsubsection {Diagrama de causa e efeito}
					\subsubsection {Confirmando a causa raiz}
					\subsubsection {Preparação para a próxima fase}
				\subsection {Melhorar}
					\subsubsection {Objetivo}
					\subsubsection {Identificando a solução}
					\subsubsection {Selecionando a solução}
					\subsubsection {Implementando a solução}
					\subsubsection {Avaliando as melhorias}
					\subsubsection {Preparação para a próxima fase}
				\subsection {Controlar}
					\subsubsection {Objetivo}
					\subsubsection {Padronizando e documentando}
					\subsubsection {Monitorando o processo}
			\section {DMADV}
				\subsection {Definir}
				\subsection {Medir}
				\subsection {Analisar}
				\subsection {Projetar}
				\subsection {Verificar}
			\section {DMAIC vs DMADV}
	
	\chapter {Conclusão}

\end{document}
