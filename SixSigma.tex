\documentclass{abnt}
%Arquivo com os principais pacotes usados e suas descrições.

%%%%%%%%%%%%%%%%%%%%%%%%%%%%%%%%%%%%%%%%%
% 			Idiomas e Acentos			%
%%%%%%%%%%%%%%%%%%%%%%%%%%%%%%%%%%%%%%%%%
\usepackage[brazil]{babel} % Habilita o uso do idioma português do brasil (PT-BR).
\usepackage[T1]{fontenc} 
%\usepackage{fontspec} % Habilita maior variedade de acentos. Pode ser necessario adicionar outros pacotes.
\usepackage{lmodern} % Habilita o uso da font Latin Modern.


%%%%%%%%%%%%%%%%%%%%%%%%%%%%%%%%%%%%%%%%%
% 				TABELAS					%
%%%%%%%%%%%%%%%%%%%%%%%%%%%%%%%%%%%%%%%%%
\usepackage{tabulary} % Cria tabelas mais facilmente.
\usepackage{booktabs} % Melhora o visual das tabelas.
\usepackage{table}{xcolor} % Pacote de cor pra as tabelas.
\usepackage{caption} % Melhora as legendas de imagens, tabela etc.

%%%%%%%%%%%%%%%%%%%%%%%%%%%%%%%%%%%%%%%%%
% 				IMAGENS					%
%%%%%%%%%%%%%%%%%%%%%%%%%%%%%%%%%%%%%%%%%
%\usepackage{graphicx} % Facilita a inserção de imagens.


%%%%%%%%%%%%%%%%%%%%%%%%%%%%%%%%%%%%%%%%%
% 			CÓDIGO FONTE				%
%%%%%%%%%%%%%%%%%%%%%%%%%%%%%%%%%%%%%%%%%
%Documentação de código fonte.
\usepackage{listings}


%%%%%%%%%%%%%%%%%%%%%%%%%%%%%%%%%%%%%%%%%
% 	Símbolos e Caracteres Matemáticos	%
%%%%%%%%%%%%%%%%%%%%%%%%%%%%%%%%%%%%%%%%%
\usepackage{amsmath}
\usepackage{amssymb}
\usepackage{amsfonts}
%\usepackage{mathspec} %Habilita o uso das fontes e dos caracteres matematicos.


%%%%%%%%%%%%%%%%%%%%%%%%%%%%%%%%%%%%%%%%%
%				ABNT					%
%%%%%%%%%%%%%%%%%%%%%%%%%%%%%%%%%%%%%%%%%
%\usepackage[alf]{abntcite} % Ordena as referencias em ordem alfabética.
\usepackage{url} %Facilita o uso de url. Pode-se usar o comando \url{...}.


%%%%%%%%%%%%%%%%%%%%%%%%%%%%%%%%%%%%%%%%%
% 			Configurações				%
%%%%%%%%%%%%%%%%%%%%%%%%%%%%%%%%%%%%%%%%%
\captionsetup{justification=centering,labelfont=bf} %Formata a legenda das figuras.
%\graphicspath{{../imgs/}} %Define o diretorio padrão para buscar as imagens da apresentação.  
%\setromanfont[Ligatures=TeX]{Crimson}
%\defaultfontfeatures{Scale=MatchLowercase, Mapping=tex-tex}

%%%%%%%%%%%%%%%%%%%%%%%%%%%%%%%%%%%%%%%%%
%				BEAMER					%
%%%%%%%%%%%%%%%%%%%%%%%%%%%%%%%%%%%%%%%%%
%Define algumas configurações que serão validas para todo o documento.  
\setbeamertemplate{section in toc}[sections numbered]
\setbeamertemplate{subsection in toc}[subsections numbered]
\setbeamertemplate{background canvas}[vertical shading][bottom=blue!3,top=blue!7]
\setbeamertemplate{caption}[numbered]


%%%%% Dados para criação da capa e folha de rosto %%%%
\autor{	Denis F. de Carvalho, 
		Guilherme A. de Macedo, 
		Matheus L. Domingues da Silva e 
		Victor H. Carlquist da Silva
}
\titulo{Sistema Seis Sigma de Produção}
\orientador{Avelino Bazanella Junior}
\comentario{Trabalho apresentado ao Prof. Avelino Bazanela Junior, na disciplina de Administração 
			presente no $2^{a}$ modulo do curso de Tecnologia em Análise e Desenvolvimento de Sistemas no IFSP-CJO.}
\instituicao{Instituto Federal de Educação, Ciência e Tecnologia de São Paulo -- \textit{campus} Campos do Jordão}
\local{Campos do Jordão}
\data{\today}

\begin{document}

	% Para utilizar o formato padrão de capa da ABNT, substituí o comando \maketitle pelo comando \capa.
	\capa
	
	\folhaderosto
	
	\begin{resumo}
		Este trabalho tem por objetivo mostrar e explicar o funcionamento do programa Seis Sigma. A construção 
		desse trabalho foi baseada em pesquisas em \textit{sites} especializados, gráficos e tabelas, bem como a consulta 
		de livros especializados.
	\end{resumo}

	\begin{abstract}
			This work aims to show and explain the workings of the Six Sigma program. The construction of this work 
			was based on research on specialized sites, graphs and tables, and consultation of specialized books.
	\end{abstract}
	
	\sumario
	
	\listadetabelas
	
	\listadefiguras
	
	\chapter {Introdução}
	
	Seis Sigma (six-sigma ou $\sigma$-seis) é um programa de melhoria de processo baseado numa 
	metodologia de solução de problemas composto por cinco etapas: Definição, Medição, Análise, 
	Melhoria e Controle. Em sua forma mais geral, o Seis Sigma é uma forma de avaliar os níveis 
	de produção de uma empresa. O Seis Sigma foi inicialmente desenvolvido visando a melhoria nos 
	processos de manufaturas, porém hoje é utilizado pelas empresas em quaisquer tipos de processos, 
	incluindo até os processos de TI (Tecnologia da Informação).
	
	O conceito do Seis Sigma é estabelecer uma métrica universal para medir os defeitos de um processo. 
	Essa métrica estabelece que quanto mais alto o nível de sigma, melhores serão os produtos produzidos, 
	porém quanto menor for o nível de sigma, maior será a quantidade de produtos ruins produzidos pela empresa.
	
	\chapter {História}
	
	O Seis Sigma foi inicialmente desenvolvido pela Motorola.
	
	\chapter {Níveis de Sigma}
		\section {Introdução}
			O sigma é utilizado para medir a variância de qualquer processo. Os níveis de sigma medem o desempenho do processo de uma empresa.
			Geralmente uma empresa adota níveis 3 ou 4 do sigma, que são níveis considerados normais.
			
			O sigma ($\sigma$) é calculado pela seguinte fórmula: 
			\begin{center}
			    \begin{equation}
			       Z = \frac{x - \mu}{\sigma}
		        \end{equation}
		     	\begin{itemize}
		     		\item[$x$] ponto que se deseja converter em $Z$;
		     		\item[$\mu$] média da normal original;
		     		\item[$\sigma$] desvio padrão da normal original.
		     	\end{itemize}
			\end{center}
			
			 
		\section {Níveis de Qualidade}
		    Cada nível de sigma possuí um limite de desvios (problemas), que é medido em  \textit{Problemas por Milhão} (PPM) 
			\begin{itemize}
			    \item {1 Sigma}
			        \subitem O 1 Sigma tolera até 697700 (PPM), possuindo um fator de sucesso do processo de 30,23\%. 
			    \item {2 Sigma}
			        \subitem O 2 Sigma tolera até 308700 (PPM), possuindo um fator de sucesso do processo de 69,13\%.
			    \item {3 Sigma}
			        \subitem O 3 Sigma tolera até 66810 (PPM), possuindo um fator de sucesso do processo de 93,32\%.
			    \item {4 Sigma}
			        \subitem O 4 Sigma tolera até 6210 (PPM), possuindo um fator de sucesso do processo de 99,379\%.
			    \item {5 Sigma}
			        \subitem O 5 Sigma tolera até 233 (PPM), possuindo um fator de sucesso do processo de 99,9767\%.
			    \item {6 Sigma}
			        \subitem O 6 Sigma tolera até 3,4 (PPM), possuindo um fator de sucesso do processo de 99,99966\%.
			\end{itemize}
			
			Dessa forma os níveis de qualidade do sigma podem ser melhores visualizados com base na tabela a seguir:
			\begin{table}[h]
				\centering
				\rowcolors{2}{gray!10}{white}

				\begin{tabular}{rcr}
					\toprule
					Sigma & Problemas por Milhão (PPM) & Porcentagem (\%) \\
					\midrule
					 1 & 697700	& 30,23 	\\
					 2 & 308700 & 69,13 	\\
					 3 & 66810 	& 93,32 	\\
					 4 & 6210 	& 99,379 	\\
					 5 & 233 	& 99,9767 	\\
					 6 & 3,4 	& 99,99966 	\\
					\bottomrule		
				\end{tabular}
	
				\label{tab_niveisSigma}
				\caption{Níveis de Sigma.}
				
			\end{table}

		\section {Curva Normal} 
		    
		\section {Vantagens do 6 Sigma}
				
	\chapter {Metodologias}
		\section {Introdução}
			\section {DMAIC}
				\subsection {Definir}
					\subsubsection {Objetivo}
					\subsubsection {Entendendo o processo}
					\subsubsection {Preparação para a próxima fase}
				\subsection {Medir}
					\subsubsection {Objetivo}
					\subsubsection {Mapeando processo}
					\subsubsection {Coletando os dados}
					\subsubsection {Analisando os dados coletados}
					\subsubsection {Calculando o nível de Sigma}
					\subsubsection {Preparação para a próxima fase}
				\subsection {Analisar}
					\subsubsection {Melhorar}
					\subsubsection {Objetivo}
					\subsubsection {Achando a causa raiz}
					\subsubsection {Diagrama de causa e efeito}
					\subsubsection {Confirmando a causa raiz}
					\subsubsection {Preparação para a próxima fase}
				\subsection {Melhorar}
					\subsubsection {Objetivo}
					\subsubsection {Identificando a solução}
					\subsubsection {Selecionando a solução}
					\subsubsection {Implementando a solução}
					\subsubsection {Avaliando as melhorias}
					\subsubsection {Preparação para a próxima fase}
				\subsection {Controlar}
					\subsubsection {Objetivo}
					\subsubsection {Padronizando e documentando}
					\subsubsection {Monitorando o processo}
			\section {DMADV}
				\subsection {Definir}
				\subsection {Medir}
				\subsection {Analisar}
				\subsection {Projetar}
				\subsection {Verificar}
			\section {DMAIC vs DMADV}
	
	\chapter {Conclusão}
	
	%\chapter{Referências Bibliográficas}
	\begin{thebibliography}{99}
		\bibitem{teste} teste
	\end{thebibliography}

\end{document}
