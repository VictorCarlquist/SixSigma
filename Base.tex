%Arquivo com os principais pacotes usados e suas descrições.

%Hbilita o uso do idioma português do brasil (PT-BR).
%\usepackage[brazilian]{babel}
\usepackage[brazil]{babel}    % Victor : funciona no meu latex =)
\usepackage[T1]{fontenc}
%\usepackage[utf8]{inputenc}   % Victor : funciona no meu latex =)
\usepackage[alf]{abntcite}

%Habilita o uso da font Latin Modern.
\usepackage{lmodern}
%Facilita o uso de url. Pode-se usar o comando \url{...}.
\usepackage{url}
%Facilita a inserção de imagens.
\usepackage{graphicx}
%Melhora as legendas de imagens, tabela etc.
\usepackage{caption}
%Documentação de código fonte.
\usepackage{listings}
%Para uso dos simbolos matemáticos.
\usepackage{amsmath}
\usepackage{amssymb}
\usepackage{amsfonts}
   
%Formata a legenda das figuras.
\captionsetup{justification=centering,labelfont=bf}

%Define algumas configurações que serão validas para todo o documento.  
%\setbeamertemplate{section in toc}[sections numbered]
%\setbeamertemplate{subsection in toc}[subsections numbered]
%\setbeamertemplate{background canvas}[vertical shading][bottom=blue!3,top=blue!7]
%\setbeamertemplate{caption}[numbered]

%Define o diretorio padrão para buscar as imagens da apresentação.  
\graphicspath{{../imgs/}}
