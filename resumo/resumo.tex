\documentclass{abnt}

%Arquivo com os principais pacotes usados e suas descrições.

%%%%%%%%%%%%%%%%%%%%%%%%%%%%%%%%%%%%%%%%%
% 			Idiomas e Acentos			%
%%%%%%%%%%%%%%%%%%%%%%%%%%%%%%%%%%%%%%%%%
\usepackage[brazilian]{babel} % Habilita o uso do idioma português do brasil (PT-BR).
\usepackage[T1]{fontenc} 
%\usepackage{fontspec} % Habilita maior variedade de acentos. Pode ser necessario adicionar outros pacotes.
\usepackage{lmodern} % Habilita o uso da font Latin Modern.


%%%%%%%%%%%%%%%%%%%%%%%%%%%%%%%%%%%%%%%%%
% 				TABELAS					%
%%%%%%%%%%%%%%%%%%%%%%%%%%%%%%%%%%%%%%%%%
\usepackage{tabulary} % Cria tabelas mais facilmente.
\usepackage{booktabs} % Melhora o visual das tabelas.
\usepackage[table]{xcolor} % Pacote de cor pra as tabelas.
\usepackage{caption} % Melhora as legendas de imagens, tabela etc.

%%%%%%%%%%%%%%%%%%%%%%%%%%%%%%%%%%%%%%%%%
% 				IMAGENS					%
%%%%%%%%%%%%%%%%%%%%%%%%%%%%%%%%%%%%%%%%%
%\usepackage{graphicx} % Facilita a inserção de imagens.


%%%%%%%%%%%%%%%%%%%%%%%%%%%%%%%%%%%%%%%%%
% 			CÓDIGO FONTE				%
%%%%%%%%%%%%%%%%%%%%%%%%%%%%%%%%%%%%%%%%%

%Documentação de código fonte.
\usepackage{listings}


%%%%%%%%%%%%%%%%%%%%%%%%%%%%%%%%%%%%%%%%%
% 	Símbolos e Caracteres Matemáticos	%
%%%%%%%%%%%%%%%%%%%%%%%%%%%%%%%%%%%%%%%%%
\usepackage{amsmath}
\usepackage{amssymb}
\usepackage{amsfonts}
%\usepackage{mathspec} %Habilita o uso das fontes e dos caracteres matematicos.


%%%%%%%%%%%%%%%%%%%%%%%%%%%%%%%%%%%%%%%%%
%				ABNT					%
%%%%%%%%%%%%%%%%%%%%%%%%%%%%%%%%%%%%%%%%%
\usepackage[alf]{abntcite} % Ordena as referencias em ordem alfabética.
\usepackage{url} %Facilita o uso de url. Pode-se usar o comando \url{...}.


%%%%%%%%%%%%%%%%%%%%%%%%%%%%%%%%%%%%%%%%%
% 			Configurações				%
%%%%%%%%%%%%%%%%%%%%%%%%%%%%%%%%%%%%%%%%%
\captionsetup{justification=centering,labelfont=bf} %Formata a legenda das figuras.
%\graphicspath{{../imgs/}} %Define o diretorio padrão para buscar as imagens da apresentação.  
%\setromanfont[Ligatures=TeX]{Crimson}
%\defaultfontfeatures{Scale=MatchLowercase, Mapping=tex-tex}

%%%%%%%%%%%%%%%%%%%%%%%%%%%%%%%%%%%%%%%%%
%				BEAMER					%
%%%%%%%%%%%%%%%%%%%%%%%%%%%%%%%%%%%%%%%%%
%Define algumas configurações que serão validas para todo o documento.  
%\setbeamertemplate{section in toc}[sections numbered]
%\setbeamertemplate{subsection in toc}[subsections numbered]
%\setbeamertemplate{background canvas}[vertical shading][bottom=blue!3,top=blue!7]
%\setbeamertemplate{caption}[numbered]

\autor{	Denis F. de Carvalho, \\
		Guilherme A. de Macedo, \\
		Matheus L. Domingues da Silva e\\ 
		Victor H. Carlquist da Silva
}
\titulo{\textbf{Resumo:} Sistema Seis Sigma de Produção}
\orientador{Avelino Bazanella Junior}
\comentario{Trabalho apresentado ao Prof. Avelino Bazanela Junior, na disciplina de Administração 
			presente no $2^{a}$ modulo do curso de Tecnologia em Análise e Desenvolvimento de Sistemas no IFSP-CJO.}
\instituicao{Instituto Federal de Educação, Ciência e Tecnologia de São Paulo -- \textit{campus} Campos do Jordão}
\local{Campos do Jordão}
\data{\today}

\begin{document}
    \maketitle
    
    \section*{Seis Sigma (6 $\sigma$)}
    
    Seis Sigma (six-sigma ou $\sigma$-seis) é um programa de melhoria de processo baseado numa 
	metodologia de solução de problemas composto por cinco etapas: Definição, Medição, Análise, 
	Melhoria e Controle. Em sua forma mais geral, o Seis Sigma é uma forma de avaliar os níveis 
	de produção de uma empresa. O Seis Sigma foi inicialmente desenvolvido visando a melhoria nos 
	processos de manufaturas, porém hoje é utilizado pelas empresas em quaisquer tipos de processos, 
	incluindo até os processos de TI (Tecnologia da Informação).
	
    A empresa  possui um nível de sigma que indica o aproveitamento dos processos e/ou dos produtos. Quanto maior
    o nível de sigma melhor é o processo da empresa.

    O 6 sigma tolera até 3,5 problemas por milhão. Mas, geralmente, a maior parte das empresas trabalham com níveis 3 ou 4 de sigma.    
    Para conseguir um bom nível de qualidade, o Seis Sigma possui dois métodos, o DMAIC e o DMADV. Esses métodos são a base do Seis Sigma.
   
    Níveis de Sigma:
    \begin{table}[h]
				\centering
				\rowcolors{2}{gray!10}{white}

				\begin{tabular}{rcr}
					\toprule
					Sigma & Problemas por Milhão (PPM) & Porcentagem (\%) \\
					\midrule
					 1 & 697700	& 30,23 	\\
					 2 & 308700 & 69,13 	\\
					 3 & 66810 	& 93,32 	\\
					 4 & 6210 	& 99,379 	\\
					 5 & 233 	& 99,9767 	\\
					 6 & 3,4 	& 99,99966 	\\
					\bottomrule		
				\end{tabular}
	
				\label{tab_niveisSigma}
				\caption{Níveis de Sigma.}
				
			\end{table}
   
\end{document}
